\documentclass[11pt]{article}
\usepackage[margin=1in]{geometry}
\usepackage{amsfonts,amsmath,amssymb}
\usepackage[none]{hyphenat}
\usepackage{fancyhdr}
\usepackage{graphicx}
\usepackage{float}
\title{Derivation of SNE Gradient}
\author{Zein\\GitHub@zein0115}
\date{July 29, 2020}

\begin{document}
\maketitle

\section{Requirements}
Basic knowledge of matrix calculus and multivariable calculus.\\
Newton's method.

\section{Notations}
The notations are almost same as article by Aapo Hyvärinen.
\subsection{Datasets}
The original dataset $\mathbf{s}$ with dimension $d_s$
$$
\mathbf{s} = 
\left[
    \begin{matrix}
        | & | &  \cdots & | \\
        s^{(1)} & s^{(2)} &  \cdots & s^{(n)}\\
        | & | &  \cdots & |
    \end{matrix}
\right] = 
\left[
    \begin{matrix}
        s_1\\s_2\\\vdots\\s_{d_s}
    \end{matrix}
\right]
\in \mathbb{R}^{d_s\times n}
$$
The mixed dataset $\mathbf{x}$ with dimension $d_x$, which is a linear combination of $\mathbf{s}$
$$
\mathbf{x} = 
\left[
    \begin{matrix}
        | & | &  \cdots & | \\
        x^{(1)} & x^{(2)} &  \cdots & x^{(n)}\\
        | & | &  \cdots & |
    \end{matrix}
\right]
\in \mathbb{R}^{d_x\times n}
$$
\subsection{Transforming matrices}
Since $\mathbf{x}$ is a linear combination of $\mathbf{s}$, assume that 
$$
A\mathbf{s} = \mathbf{x}
$$
Where
$$
    A\in\mathbb{R}^{d_x\times d_s}
$$
Assume $W = A^{-1}$ exists, then
$$
\mathbf{s} = W\mathbf{x}
$$
If we let $w_i\in\mathbb{R}^{d_x}$ be a vector which satisfy
$$
W = \left[
    \begin{matrix}
        w_1^T\\w_2^T\\\vdots\\w_{d_s}^T
    \end{matrix}
\right]
$$
Then
$$
s_i = w_i^T\mathbf{x}
$$
Alternatively, if we just use $w^T$ to express one row of $W$, then
$$
y = w^T\mathbf{x}
$$
Implies $y=s_i$ for some $i$.
\subsection{Preprocessing}
Before doing FastICA, dataset $\mathbf{x}$ need to be centerning and whitening to simplify the calculation
$$
\begin{aligned}
    &\mathbf{x} = \mathbf{x} - E[\mathbf{x}]\\
    &\tilde{\mathbf{x}} = ED^{-1/2}E^T\mathbf{x}
\end{aligned}
$$
Where $E$ and $D$ are generate by eigonvalue decomposition of the covariance matrix, $D^{-1/2}=\sqrt{D^{-1}}$
$$
E[\mathbf{x}\mathbf{x}^T] = EDE^T
$$
After centerning and whitening ($\mathbf{x}$ is used to denote $\tilde{\mathbf{x}}z$)
$$
\begin{aligned}
    &E[\mathbf{x}] = 0\\
    &E[\mathbf{x}\mathbf{x}^T] = I
\end{aligned}
$$
\subsection{Optimization}
The goal of FastICA for one unit is to find a vector $w$ that maximize the nongaussianity of $w^Tx$, where
Nongaussianity is a measurement of independence. \\
Nongaussianity is measured by the approximation of negentropy $J(w^T\mathbf{x})$. 
More negentropy, more independence.
$$
J(w^T\mathbf{x}) \propto \{E[G(w^T\mathbf{x})]-E[G(\nu)]\}^2
$$
Where $\nu$ is a Gaussian variable of zero mean and unit variance, while $G$ is a nonquadratic function.
For example
$$
G_1(u)=\frac{1}{a_1} \log\cosh a_1u, G_2(u)=-\exp(-u^2/2)
$$
\subsection{Special note}
I am going to introduce a matrix product rule called "$*$". Which is same as ".*" in MATLAB and "*" in Numpy
for matrix calculation, namely, element product. Following are some examples
$$
\left[
    \begin{matrix}
        a \\ b \\ c
    \end{matrix}
    \right]
    *
    \left[
        \begin{matrix}
            d \\ e \\ f
        \end{matrix}
        \right]
        =
    \left[
        \begin{matrix}
            ad \\ be \\ cf
        \end{matrix}
    \right],
\left[
    \begin{matrix}
        a \\ b \\ c
    \end{matrix}
    \right]
    *
    \left[
        \begin{matrix}
            d & g\\ 
            e & h\\ 
            f & i
        \end{matrix}
        \right]
        =
        \left[
            \begin{matrix}
                ad & ag\\ 
                be & bh\\ 
                cf & ci
            \end{matrix}
            \right]
$$
$$
\left[
    \begin{matrix}
        a & b & c
    \end{matrix}
\right] * 
\left[
    \begin{matrix}
        d & e & f\\
        g & h & i
    \end{matrix}
\right] = 
\left[
    \begin{matrix}
        ad & be & cf\\
        ag & bh & ci
    \end{matrix}
\right]
$$
Intuitively, $(A*B)^T = A^T*B^T$.

\section{Derivation of FastICA for one unit}
To maximize $J(w^T\mathbf{x})$, we need to find the optima of $E[G(w^T\mathbf{x})]$. \\
To simplify the calculation, let $\|w\|=1$, i.e., $w^Tw=1$.\\\\
Then our goal is to find the $w$ that maximize $E[G(w^T\mathbf{x})]$ constrained by $\|w\|=1$, 
according to Kuhn-Tucker conditions, we need to find $w$ satisfy 
\begin{equation}
    \frac{\partial E[G(w^T\mathbf{x})]}{\partial w} = \lambda\frac{\partial w^Tw}{\partial w}
\end{equation}
Let $g=G'$ and $\beta = 2\lambda$, where $\lambda$ is Lagrange multiplier. 
The left hand side equals to
$$
\begin{aligned}
    \frac{\partial E[G(w^T\mathbf{x})]}{\partial w} &= 
    \frac{\partial}{\partial w}\frac{1}{n}\sum_{i=1}^nG(w^Tx^{(i)})\\
    &=\frac{1}{n}\sum_{i=1}^n \frac{\partial}{\partial w} G(w^Tx^{(i)})\\
    &=\frac{1}{n}\sum_{i=1}^n \frac{\partial w^Tx^{(i)}}{\partial w} \frac{\partial}{\partial w^Tx^{(i)}} G(w^Tx^{(i)})\\
    &= \frac{1}{n}\sum_{i=1}^n x^{(i)T}g(w^Tx^{(i)})\\
    &= \left[ \frac{1}{n}\sum_{i=1}^n x^{(i)}g(w^Tx^{(i)}) \right]^T\\
    &=E\left[
        \mathbf{x} * g(w^T\mathbf{x})
    \right]^T\\
\end{aligned}
$$
The right hand side is
$$
\lambda\frac{\partial w^Tw}{\partial w} = 2\lambda w^T = \beta w^T
$$
Thus the euqation (1) could be rewrite to 
\begin{equation}
    E\left[
        \mathbf{x} * g(w^T\mathbf{x})
    \right]^T - \beta w^T = 0
\end{equation}
Equivalent to 
\begin{equation}
    E\left[
        \mathbf{x} * g(w^T\mathbf{x})
    \right] - \beta w = 0
\end{equation}
FastICA uses Newton's method to solve equation (3). Newton's method is a numerical method 
to find root of a nonlinear equation. For a nonlinear equation $F(x)=0$, do iteration
$$
x = x - \frac{F(x)}{F'(x)}
$$
If $F$ is a vector-valued multivariable function and $x$ is a vector, do
$$
x = x - J_F^{-1}(x)F(x)
$$
Where $J_F(x) = F'(x)$ is Jacobian matrix while $J_F^{-1}(x)$ is the inverse.\\\\
Consider $F(w) = E\left[\mathbf{x} * g(w^T\mathbf{x})\right] - \beta w$, then we are going to solve
\begin{equation}
    F(w) = 0
\end{equation}
First find the Jacobian matrix
$$
\begin{aligned}
    J_F(w) &= \frac{\partial F(w)}{\partial w}\\
    &= \frac{\partial }{\partial w}E\left[\mathbf{x} * g(w^T\mathbf{x})\right] - 
    \frac{\partial }{\partial w}\beta w\\
    &= \frac{\partial }{\partial w}\frac{1}{n}\sum_{i=1}^n x^{(i)}g(w^Tx^{(i)}) - \beta I\\
    &= \frac{1}{n}\sum_{i=1}^n x^{(i)}x^{(i)T}g'(w^Tx^{(i)}) - \beta I
\end{aligned}
$$
To simplify the calculation, we assume that $x^{(i)}x^{(i)T}$ and $g'(w^Tx^{(i)})$ are independent variables, 
then the approximation is given as
$$
\begin{aligned}
    J_F(w) &= \frac{1}{n}\sum_{i=1}^n x^{(i)}x^{(i)T}g'(w^Tx^{(i)}) - \beta I\\
    &\approx \frac{1}{n}\sum_{i=1}^n x^{(i)}x^{(i)T} \frac{1}{n}\sum_{i=1}^n g'(w^Tx^{(i)}) - \beta I\\
    &= \frac{1}{n}\mathbf{x}\mathbf{x}^T\frac{1}{n}\sum_{i=1}^n g'(w^Tx^{(i)}) - \beta I\\
    &= E[\mathbf{x}\mathbf{x}^T]E[g'(w^Tx^{(i)})] - \beta I\\
    &= IE[g'(w^Tx^{(i)})] - \beta I\\
    &= I\left\{E[g'(w^Tx^{(i)})]-\beta\right\}
\end{aligned}
$$
Then $J_F(w)$ is a diagonal matrix, the inverse could be simply find as $J_F^{-1}=I\left\{E[g'(w^Tx^{(i)})]-\beta\right\}^{-1}$.
Apply $J_F^{-1}(w)$ to the iteration euqation $w = w - J_F^{-1}(w)F(w)$
$$
\begin{aligned}
    w &= w - J_F^{-1}F(w)\\
    &=w - I\frac{E\left[\mathbf{x} * g(w^T\mathbf{x})\right] - \beta w}
    {E[g'(w^Tx^{(i)})]-\beta}\\
    &= \frac{1}{E[g'(w^Tx^{(i)})]-\beta} \left\{ 
        E[g'(w^Tx^{(i)})]w - \beta w - E[x * g(w^Tx^{(i)})] + \beta w
    \right\}\\
    &= \frac{1}{E[g'(w^Tx^{(i)})]-\beta}\left\{ 
        E[g'(w^Tx^{(i)})]w- E[x * g(w^Tx^{(i)})]
    \right\}
\end{aligned}
$$
Since $E[g'(w^Tx^{(i)})]-\beta$ is a scaler and $w$ is constrained as $\|w\|=1$, 
$E[g'(w^Tx^{(i)})]-\beta$ could be ignored by applying
\begin{align}
    w &= E[g'(w^Tx^{(i)})]w- E[x * g(w^Tx^{(i)})]\\
    w &= \frac{w}{\|w\|}
\end{align}
in each iteration, which is a basic form of FastICA. \\\\
I would not post the derivation of FastICA for several unit here, since the derivation in the original
article is comprehensive enough. 

\renewcommand\refname{Reference}
\nocite{1}
\nocite{*} 
\bibliographystyle{plain}
\bibliography{reference.bib}

\end{document}