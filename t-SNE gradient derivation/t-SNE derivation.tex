\documentclass[11pt]{article}
\usepackage[margin=1in]{geometry}
\usepackage{amsfonts,amsmath,amssymb}
\usepackage[none]{hyphenat}
\usepackage{fancyhdr}
\usepackage{graphicx}
\usepackage{float}
\title{Derivation of SNE Gradient}
\author{Zein\\GitHub@zein0115}
\date{July 16, 2020}

\begin{document}
\maketitle

\section{Brief Review\cite{maaten2008visualizing}}
For a given high-dimensional dataset $X = \{x_1, x_2, \dots, x_n\}$, $x_i\in \mathbb{R} ^D$, define
$$
    p_{j|i} = \frac{\exp(-\| x_i - x_j \|^2/2\sigma_i^2)}{\sum_{k\neq i}
    {\exp(-\| x_i - x_k \|^2/2\sigma_i^2)}}
$$
For a randomized reduced dataset $Y = \{y_1, y_2, \dots, y_n\}$, $y_i\in \mathbb{R} ^M$, $M<D$, in SNE, define
$$
    q_{j|i}  = \frac{\exp(-\| y_i - y_j \|^2)}{\sum_{k\neq i}{\exp(-\| y_i - y_k \|^2)}}
$$
In t-SNE, $q_{ij}$ is defined as
$$
    q_{ij} = \frac{(1+\| y_i - y_j \|^2)^{-1}}{\sum_{k\neq l}{(1+\| y_l - y_k \|^2)^{-1}}}
$$
Define the cost function in SNE
$$
    C = \sum_{i}KL(P_i \| Q_i) = \sum_i \sum_j p_{j|i} \log\frac{p_{j|i}}{q_{j|i}}
$$
In t-SNE, the cost function is
$$
    C = \sum_{i}KL(P_i \| Q_i) = \sum_i \sum_j p_{ij} \log\frac{p_{ij}}{q_{ij}}
$$
Where
$$
    p_{ij} = \frac{p_{j|i} + p_{i|j}}{2N}
    % , q_{ij} = \frac{\exp(-\| y_i - y_j \|^2)}
    % {\sum_{k\neq l}{\exp(-\| y_l - y_k \|^2)}} \ \ {\rm or} \ \ 
    % \frac{(1+\| y_i - y_j \|^2)^{-1}}{\sum_{k\neq l}{(1+\| y_l - y_k \|^2)^{-1}}}
$$
Using gradient descent to optimize $Y$, in each iteration, apply
$$
    y_i^{(t+1)} = y_i^{(t)} - \eta\frac{\partial{C}}{\partial{y_i}}
$$
Which $\eta$ is learning rate.

\section{t-SNE}
It might be a little complicated to find $\partial{C}/\partial{y_i}$ directly, 
so I define two auxiliary variables $d_{ij}$ and $Z$ here. \cite{maaten2008visualizing}
$$
    d_{kl} = \|y_k - y_l\|, Z = \sum_{k\neq l} (1+\|y_l - y_k\|^2)^{-1}
$$
The basic idea to find the gradient is using the chain rule
$$
\frac{\partial{f}}{\partial{x}} = \frac{\partial{f}}{\partial{u}}\frac{\partial{u}}{\partial{x}} + 
\frac{\partial{f}}{\partial{v}}\frac{\partial{v}}{\partial{x}}
$$
To simplify the derivation, we regarding $C$ as the function of $d_{ij}$, $d_{ij}$ as the function of $y_i$. 
Then, the gradient for $i^{th}$ variable is
$$
\frac{\partial{C}}{\partial{y_i}} = \sum_{k}\sum_{l} \frac{\partial{C}}{\partial{d_{kl}}}
\frac{\partial{d_{kl}}}{\partial{y_i}}
$$
It could be easily derived that only $\partial d_{ij}/\partial y_i$ and $\partial d_{ji}/\partial y_i$, $i\neq j$, 
are nonzero, thus
$$
\frac{\partial{C}}{\partial{y_i}} = \sum_j \frac{\partial{C}}{\partial{d_{ij}}}
\frac{\partial{d_{ij}}}{\partial{y_i}} +\frac{\partial{C}}{\partial{d_{ji}}}
\frac{\partial{d_{ji}}}{\partial{y_i}}
$$
% Since $d_{ij}=d_{ji}$, 
% $$
% \frac{\partial{C}}{\partial{y_i}} = 2\sum_j \frac{\partial{C}}{\partial{d_{ij}}}
% \frac{\partial{d_{ij}}}{\partial{y_i}}
% $$
Find $\partial{d_{ij}}/\partial{y_i}$, 
$$
\frac{\partial{d_{ij}}}{\partial{y_i}} = \frac{\partial{\|y_i - y_j\|}}{\partial{y_i}} = 
\frac{\partial\sqrt{(y_i-y_j)^{\rm T}(y_i-y_j)}}{\partial{y_i}} = \frac{y_i-y_j}
{\sqrt{(y_i-y_j)^{\rm T}(y_i-y_j)}} = \frac{y_i-y_j}{d_{ij}}
$$
(Note: according to matrix calculus, the result should be $\displaystyle\frac{(y_i-y_j)^{\rm T}}{d_{ij}}$.
Since we are going to find gradient, I will use $\displaystyle\frac{y_i-y_j}{d_{ij}}$ there.) \\
Find $\partial C/\partial d_{ij}$
$$
\frac{\partial C}{\partial d_{ij}} = \frac{\sum_k \sum_l p_{kl} \log\frac{p_{kl}}{q_{kl}}}
{\partial d_{ij}} = \frac{\sum_k\sum_l p_{kl}(\log p_{kl} - \log q_{kl})}{\partial d_{ij}}
$$
Since $p_{ij}$ is a constant and $p_{ii}=0$
$$
\frac{\partial C}{\partial d_{ij}} = \frac{\partial{\sum_k\sum_l -p_{kl}\log q_{kl}}}{\partial d_{ij}} = 
\frac{\partial\sum_{k\neq l}-p_{kl}\log q_{kl}}{\partial d_{ij}}
$$
Using $d_{ij}$ and $Z$, $q_{ij}$ could be expressed as
$$
    q_{ij} = \frac{(1+d_{ij}^2)^{-1}}{Z}
$$
Then $\partial C/\partial d_{ij}$ could be expressed by
$$
\frac{\partial C}{\partial d_{ij}} = \frac{\partial \sum_{k\neq l}-p_{kl}\log(1+d_{kl}^2)^{-1}}{\partial d_{ij}} + 
\frac{\partial \sum_{k\neq l} p_{kl}\log Z}{\partial d_{ij}}
$$
Since $C$ is a function of $d_{ij}$, $\forall (i, j)\neq (k, l)$, $\partial d_{kl}/\partial d_{ij}=0$, the first term could be eliminate to 
$$
\frac{\partial \sum_{k\neq l}-p_{kl}\log(1+d_{kl}^2)^{-1}}{\partial d_{ij}} = -p_{ij}\frac{\partial\log (1+d_{ij}^2)^{-1}}{\partial d_{ij}} = 
2p_{ij}d_{ij}(1+d_{ij}^2)^{-1}
$$
Notice that $\sum_{k\neq l}p_{kl}=1$, the second term could be eliminate to
$$
\begin{aligned}
    \frac{\partial \sum_{k\neq l} p_{kl}\log Z}{\partial d_{ij}} &= \sum_{k\neq l}p_{kl}\frac{1}{Z}\frac{\partial Z}{\partial d_{ij}}
    = \sum_{k\neq l}-2p_{kl}\frac{1}{Z}d_{ij}(1+d_{ij}^2)^{-2}\\ &= \sum_{k\neq l} -2p_{kl}q_{ij}d_{ij}(1+d_{ij}^2)^{-1} = -2q_{ij}d_{ij}(1+d_{ij}^2)^{-1}
\end{aligned}
$$
Therefore $\partial C/\partial d_{ij}$ equals to 
$$
\frac{\partial C}{\partial d_{ij}} = 2p_{ij}d_{ij}(1+d_{ij}^2)^{-1} - 2q_{ij}d_{ij}(1+d_{ij}^2)^{-1} = 
2d_{ij}(1+d_{ij}^2)^{-1}(p_{ij}-q_{ij})
$$
Then the $\partial C/\partial d_{ji}$ could also be derived using the same way
$$
\frac{\partial C}{\partial d_{ji}} = 2d_{ji}(1+d_{ji}^2)^{-1}(p_{ji}-q_{ji})
$$
Combine $\partial{d_{ij}}/\partial{y_i}$, $\partial C/\partial d_{ij}$ and $\partial C/\partial d_{ij}$, since $p_{ij}=p_{ji}$ and 
$q_{ij}=q_{ji}$, for $y_i$, $d_{ij}=d_{ji}$
$$
\begin{aligned}
    \frac{\partial C}{\partial y_i} &= \sum_j \frac{y_i-y_j}{d_{ij}}2d_{ij}(1+d_{ij}^2)^{-1}(p_{ij}-q_{ij}) + 
    \sum_j \frac{y_j-y_i}{d_{ji}}2d_{ji}(1+d_{ji}^2)^{-1}(p_{ji}-q_{ji})\\
    & = 
    4\sum_j (p_{ij}-q_{ij})(1+\|y_i-y_j\|^2)^{-1}(y_i-y_j)
\end{aligned}
$$

\section{SNE}
Same as t-SNE, we define some auxiliary variables at first
\cite{maaten2008visualizing}
$$
    d_{kl} = \|y_k - y_l\|, Z_{i} = \sum_{k\neq i} \exp (-\|y_i - y_k\|^2) = 
    \sum_{k\neq i}\exp (-d_{ik}^2)
$$
Using the same logic in t-SNE
$$
\frac{\partial{C}}{\partial{y_i}} = \sum_{k}\sum_{l} \frac{\partial{C}}{\partial{d_{kl}}}
\frac{\partial{d_{kl}}}{\partial{y_i}}
$$
It could be easily derived that only $\partial d_{ij}/\partial y_i$ and $\partial d_{ji}/\partial y_i$, $i\neq j$, 
are nonzero, thus
$$
\frac{\partial{C}}{\partial{y_i}} = \sum_j \frac{\partial{C}}{\partial{d_{ij}}}
\frac{\partial{d_{ij}}}{\partial{y_i}} +\frac{\partial{C}}{\partial{d_{ji}}}
\frac{\partial{d_{ji}}}{\partial{y_i}}
$$
% Since $d_{ij}=d_{ji}$, 
% $$
% \frac{\partial{C}}{\partial{y_i}} = 2\sum_j \frac{\partial{C}}{\partial{d_{ij}}}
% \frac{\partial{d_{ij}}}{\partial{y_i}}
% $$
Find $\partial{d_{ij}}/\partial{y_i}$, 
$$
\frac{\partial{d_{ij}}}{\partial{y_i}} = \frac{\partial{\|y_i - y_j\|}}{\partial{y_i}} = 
\frac{\partial\sqrt{(y_i-y_j)^{\rm T}(y_i-y_j)}}{\partial{y_i}} = \frac{y_i-y_j}
{\sqrt{(y_i-y_j)^{\rm T}(y_i-y_j)}} = \frac{y_i-y_j}{d_{ij}}
$$
Find $\partial C/\partial d_{ij}$
$$
\frac{\partial C}{\partial d_{ij}} = \frac{\sum_k \sum_l p_{l|k} \log\frac{p_{l|k}}{q_{l|k}}}
{\partial d_{ij}} = \frac{\sum_i\sum_j p_{j|i}(\log p_{j|i} - \log q_{j|i})}{\partial d_{ij}}
$$
Since $p_{j|i}$ is a constant and $p_{i|i}=0$
$$
\frac{\partial C}{\partial d_{ij}} = \frac{\partial{\sum_k\sum_l -p_{l|k}\log q_{j|i}}}{\partial d_{ij}} = 
\frac{\partial\sum_{k\neq l}-p_{l|k}\log q_{l|k}}{\partial d_{ij}}
$$
Using $d_{ij}$ and $Z_i$, $q_{j|i}$ could be expressed as
$$
    q_{j|i} = \frac{\exp(-d_{ij}^2)}{Z_i}
$$
Then $\partial C/\partial d_{ij}$ could be expressed by
$$
\frac{\partial C}{\partial d_{ij}} = \frac{\partial \sum_{k\neq l}-p_{l|k}\log(\exp (-d_{kl}^2))}{\partial d_{ij}} + 
\frac{\partial \sum_{k\neq l} p_{l|k}\log Z_k}{\partial d_{ij}}
$$
Since $C$ is a function of $d_{ij}$, $\forall (i, j)\neq (k, l)$, $\partial d_{kl}/\partial d_{ij}=0$, 
also $p_{j|i}$ is a constant, the first term could be eliminate to 
$$
\frac{\partial \sum_{k\neq l}-p_{l|k}\log(\exp (-d_{kl}^2))}{\partial d_{ij}} = 
\frac{\partial \sum_{k\neq l}p_{l|k}d_{kl}^2}{\partial d_{ij}}=
\frac{\partial p_{j|i}d_{ij}^2}{\partial d_{ij}} = 2p_{j|i}d_{ij}
$$
Now we consider the second term. Obviously, only $\partial Z_i/\partial d_{ij}$ is nonzero term, thus
$$
\begin{aligned}
    \frac{\partial \sum_{k\neq l} p_{l|k}\log Z_k}{\partial d_{ij}} &= 
    \frac{\partial\sum_{l\neq i}p_{l|i}\log Z_i}{\partial d_{ij}}\\ 
    &= \sum_{l\neq i}p_{l|i}\frac{1}{Z_i}\frac{\partial Z_i}{\partial d_{j|i}}\\
    &= \sum_{l\neq i}p_{l|i}\frac{1}{Z_i}\left[-2d_{ij}\exp(-d_{ij}^2)\right]\\
    &= -2\sum_{l\neq i}p_{l|i}q_{j|i}d_{ij}\\
    &= -2q_{j|i}d_{ij}
\end{aligned}
$$
Combine two terms
$$
    \frac{\partial{C}}{\partial d_{ij}} = 2p_{j|i}d_{ij} - 2q_{j|i}d_{ij} = 2d_{ij}(p_{j|i}-q_{j|i})
$$
In same way, $\partial C/\partial d_{ij}$ could be expressed by
$$
    \frac{\partial C}{\partial d_{ij}} = 2d_{ji}(p_{i|j}-q_{i|j})
$$
Combine $\partial{d_{ij}}/\partial{y_i}$, $\partial C/\partial d_{ij}$ and $\partial C/\partial d_{ij}$
$$
\begin{aligned}
    \frac{\partial{C}}{\partial{y_i}} &= \sum_j \frac{\partial{C}}{\partial{d_{ij}}}
    \frac{\partial{d_{ij}}}{\partial{y_i}} +\frac{\partial{C}}{\partial{d_{ji}}}
    \frac{\partial{d_{ji}}}{\partial{y_i}}\\
    &= \sum_j 2d_{ij}(p_{j|i}-q_{j|i})\frac{y_i-y_j}{d_{ij}} + 2d_{ji}(p_{i|j}-q_{i|j})\frac{y_i-y_j}{d_{ji}}\\
    &= 2\sum_j (p_{j|i}+p_{i|j}-q_{j|i}-q_{i|j})(y_i-y_j)
\end{aligned}
$$

\renewcommand\refname{Reference}
\bibliographystyle{plain}
\bibliography{reference.bib}
\end{document}